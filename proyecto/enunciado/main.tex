\documentclass[11pt,letterpaper,spanish,oneside]{article}
\usepackage[utf8]{inputenc}
\usepackage[spanish, es-tabla]{babel}
\usepackage[right=1.8cm,left=2.5cm,top=2cm,bottom=2cm,headsep=1cm,footskip=1cm]{geometry}
\usepackage{graphicx,amssymb,amsmath,enumerate,verbatim, tabularx, subcaption, enumitem}
\usepackage{hyperref}
\hypersetup{
    colorlinks=true,
    linkcolor=blue,
    filecolor=magenta,      
    urlcolor=cyan,
}

\urlstyle{same}

%\usepackage[caption=false,font=footnotesize]{subfig}
\renewcommand{\tablename}{Tabla}
\decimalpoint
\parskip=0.5\baselineskip \advance\parskip by 0pt plus 2pt
\renewcommand{\baselinestretch}{1.1}
\setlength{\parindent}{0in}

\begin{document}

\pagestyle{empty}

\begin{center}
{\large \textbf{INFO239 Comunicaciones} \\ \textbf{Trabajo práctico 1}}
\end{center}

\begin{tabular}{l l}
Profesores: Christian Lazo y Pablo Huijse & \hspace{30mm} Semestre Otoño 2021 \\
%Ayudante:   & \\
\end{tabular}
\vspace{5mm}

\section*{Introducción}

La Corporación Nacional Forestal (CONAF) está desarrollando un plan para detectar tempranamente incendios forestales. Para esto se ha instalado una cámara de vigilancia en una zona estratégica del Parque Nacional Torres del Paine, Región de Magallanes. El objetivo es que la información de la cámara pueda ser visualizada de forma remota por los operadores de CONAF. \textbf{Existen dos problemáticas a considerar: (1) el ruido que afecta gravemente la calidad de las imágenes y (2) el ancho de banda limitado del enlace entre la cámara y el centro de vigilancia.}

El director regional de CONAF ha abierto una licitación para un software que reciba como entrada el flujo de imágenes en formato crudo, que elimine el ruido y que finalmente transmita un flujo codificado reteniendo la mejor calidad visual posible. La licitación considera también un segundo software ha instalarse en el centro de operaciones de CONAF que debe recibir y decodificar el flujo de datos. 

\section*{Encargo}

\begin{enumerate}
    \item Analice e identifique los tipos de ruido que afectan a las imágenes. Diseñe e implemente un algoritmo que filtre los efectos del ruido
    \item Diseñe e implemente un par transmisor/receptor para el flujo de imágenes restaurado. Defina un esquema en términos de los bloques de transformación, cuantización y codificación. Considere un codificador de largo de palabra variable. Puede omitir la etapa de codificación de canal. \textbf{Justifique adecuadamente la elección de los algoritmos y parámetros a utilizar}
    \item Evalúe el desempeño de sus algoritmos midiendo 
    \begin{enumerate}
        \item El error de distorsión entre la imagen filtrada antes de ser transmitida y la imagen decodificada por el receptor
        \item El peso digital de los cuadros codificados y la tasa de compresión, es decir el peso codificado dividido peso original
    \end{enumerate} 
    Encuentre un compromiso que permita enviar los datos en tiempo real considerando tres anchos de banda hipotéticos de 500Kbps, 1Mbps y 5Mbps. 
    \item Lea atentamente las instrucciones generales y las instrucciones de confección de informe
\end{enumerate}


\newpage

\section*{Instrucciones generales}

\begin{enumerate}[noitemsep,topsep=0pt]
    \item Implemente las funciones \texttt{denoise}, \texttt{code} y \texttt{decode} del archivo \texttt{mycodec.py}. No es valido usar codificadores/decodificadores que ya estén implementados
    \item Se usa la librería \texttt{zmq} para simular la transmisión de datos. Para probar sus algoritmos ejecute primero el script \texttt{receiver.py} y luego el script \texttt{transmiter.py}
    \item No manipule o modifique el archivo \texttt{camera.py}, todos los demás archivos pueden ser modificados
    \item Revise referencias pero no calque soluciones existentes. Se premiará su creatividad
    \item Trabaje en grupos de máximo tres estudiantes
    \item Sea leal y honesto, no copie. \href{https://www.acm.org/about-acm/code-of-ethics-in-spanish}{Aténgase al código de ética de la ACM}. Sea responsable y administre bien su tiempo
    \item Consultas al correo phuijse at inf dot uach dot cl o por slack
\end{enumerate}

\subsection*{Sobre las evaluaciones}

\begin{enumerate}[noitemsep,topsep=0pt]
    \item Informe los integrantes del grupo de trabajo por correo electrónico o slack a más tardar el día Viernes 23 de Abril de 2021
    \item Suba su proyecto a \texttt{siveducmd.uach.cl} antes de las 23:59 del 30 de Abril de 2021. No se recibirán entregas atrasadas sin justificación formal. Sólo una entrega por grupo
    \item Entregue un archivo \texttt{zip} con lo siguiente:
    \begin{enumerate}
        \item \textbf{Informe autocontenido:} Debe considerar introducción, revisión bibliográfica, metodología, resultados, conclusiones y referencias. El informe no debe superar las 10 planas en tamaño de fuente 11. Se debe citar usando el formato IEEE. Cuide su ortografía y redacción. Sólo se aceptará informes en formato PDF. Se recomienda usar \LaTeX
        \item \textbf{Presentación en video:} Los integrantes deben realizar una exposición de no más de 10 minutos presentando su diseño, implementación y análisis de resultados. La exposición debe ser grabada en formato comprimido y entregada junto al informe
        \item \textbf{Código fuente:} Adjunte todos los códigos y rutinas necesarios para ejecutar su proyecto
    \end{enumerate}

    
    
\end{enumerate}


\nocite{*}

\bibliographystyle{IEEEtran}
\bibliography{references}



\end{document}